% --------------------------------------------------------------
%                         Template
% --------------------------------------------------------------

\documentclass[10pt]{article} %draft = show box warnings
\usepackage[a4paper, total={6.5in,10.2in}]{geometry} % Flexible and complete interface to document dimensions
\usepackage[utf8]{inputenc} % Accept different input encodings [utf8]
\usepackage[T1]{fontenc}    % Standard package for selecting font encodings
\usepackage{lmodern} % Good looking T1 font\usepackage{tikz}


% --------------------------------------------------------------
%                       Packages
% --------------------------------------------------------------
\usepackage{float} % Improved interface for floating objects
\usepackage{amsmath,amsthm,amssymb} % American Mathematics Society facilities
\usepackage[linktoc=all]{hyperref} % create hyperlinks
\usepackage{graphicx,booktabs,array}
\usepackage{subcaption}
\usepackage{xcolor}
\usepackage{tikz}
\usepackage{algorithmicx}
\usepackage{algorithm}
\usepackage{algpseudocode}
\usepackage[framemethod=TikZ]{mdframed}
\usepackage{epstopdf}
% --------------------------------------------------------------
%                       Exercise Env
% --------------------------------------------------------------

\newtheoremstyle{question-style}% name of the style to be used
  {20pt}% measure of space to leave above the theorem. E.g.: 3pt
  {3pt}% measure of space to leave below the theorem. E.g.: 3pt
  {}% name of font to use in the body of the theorem
  {}% measure of space to indent
  {}% name of head font
  {}% punctuation between head and body
  { }% space after theorem head; " " = normal interword space
  {\bfseries Question \thmnumber{#2}.}% Manually specify head
  
\theoremstyle{question-style}
\newtheorem{answer}{\arabic{answer}}

% --------------------------------------------------------------
%                       Document
% --------------------------------------------------------------
\begin{document}
% --------------------------------------------------------------
%                       Header
% --------------------------------------------------------------
\noindent
\normalsize\textbf{Advanced MAchine Learning \& Deep Learning} \hfill \textbf{UPMC}\\
\normalsize\textbf{AMAL} \hfill \textbf{\today}
\flushright{\small Roger Leite Lucena -- \texttt{rogerleitelucena@gmail.com} }

{\small Alexandre Ribeiro João Macedo --  \texttt{arj.macedo@gmail.com}}\vspace{20pt}
\centerline{\Large \textbf{TD 1 - Définition de fonctions en pyTorch}}
\vspace{20pt}

% --------------------------------------------------------------
%                       Answers
% --------------------------------------------------------------

\begin{answer} % Question 1
\begin{flalign*}
    && \frac{\partial L_h \circ h}{\partial x_k}(x) = \sum_{i=1}^{p} \frac{\partial L_h }{\partial y_i}\left(h(x)\right) \frac{\partial h_i}{\partial x_k}(x) && \text{(chain rule)}
\end{flalign*}
\end{answer}

\begin{answer} \leavevmode% Question 2

\begin{itemize}
    \item \begin{align*}
        \frac{\partial g \circ \text{mse}}{\partial \textbf{y}}(\textbf{y})
    \end{align*}
\end{itemize}

\begin{itemize}
    \item \begin{align*}
            \frac{\partial g \circ \text{mse} \circ f}{\partial \mathbf{x_i}}(\mathbf{x}, \mathbf{w}, b)
    \end{align*}
\end{itemize}

\begin{itemize}
    \item \begin{align*}
        \frac{\partial g \circ \text{mse} \circ f}{\partial \mathbf{w_i}}(\mathbf{x}, \mathbf{w}, b)
    \end{align*}
\end{itemize}

\begin{itemize}
    \item \begin{align*}
        \frac{\partial g \circ \text{mse} \circ f}{\partial b}(\mathbf{x}, \mathbf{w}, b)
    \end{align*}
\end{itemize}

\end{answer} 


\end{document}
